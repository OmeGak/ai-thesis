% !TEX root = ../thesis.tex

\chapter{Related Work}
\label{sec:related}

\begin{description}
  \item[ImageNet]
  A large-scale, comprehensive and diverse database of accurately human-annotated images organized in a semantic hierarchy \cite{Deng2009} and it has become one of the main datasets in the field of object recognition.
  It it, each noun in English, coming from the lexical database WordNet \cite{Wilkniss1998}, is associated with hundreds of clean high-resolution images.
  These images depict different representations of the concepts or different perspectives and lighting conditions of the objects.
  The hierarchical structure of the database makes it possible to interlink concepts, allowing algorithms to recognizing several concepts at the same time (dog/mammal/animal).
  Its creation was motivated by the need to organize the huge amount of image data available on the Internet and make it available and useful to researchers of computer vision.

  \item[ImageNet Large-Scale Visual Recognition Challenge]
  The de-facto benchmark for large-scale object recognition algorithms \cite{Russakovsky2015}.
  It is run as a yearly competition since 2010 and has been one of the accelerators of the improvement of object recognition algorithms.
  The competition provides a publicly available dataset all algorithms have to classify to facilitate a standard evaluation between contending algorithms.
  A workshop is also organized to share results and discuss the strategies of the most accurate and innovative algorithms each year.

  \item[Deep learning framework]
  A deep learning framework is a piece of software that aims to provide a set of tools to design deep neural networks.
  They effectively separate the implementation of the network from its model.
  This has proved crucial for sharing trained models among researchers, helping the enthusiasm for deep neural networks.
  Some of the most popular ones currently are: TensorFlow, developed by Google \cite{Abadi2015}; Theano, developed by l’Université de Montréal \cite{Bergstra2010}; Caffe, developed by the Berkeley Vision and Learning Center \cite{Jia2014}; or Torch, supported by Facebook and Nvidia \cite{Collobert2002}.
\end{description}
